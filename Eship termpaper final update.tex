\documentclass[notitlepage]{report}
\usepackage[utf8]{inputenc}
\usepackage[top=6em, bottom=6em, left=1.2in, right=1.2in]{geometry}
\usepackage{ragged2e}
\date{\parbox{\linewidth}{\centering%
B.Vasudha, 19120022\endgraf
Metallurgical and materials Engg 2021-22 (II) \endgraf
National Institute of Technology, Raipur\endgraf
\begin{flushright}
Guided By:-\endgraf
Dr.Saurabh Gupta\endgraf 
\end{flushright}}}
\begin{document}

\title{\textbf{\line(1,0){425}\\Entrepreneurial education at schools-
A vision for the future ahead
\\\line(1,0){425}}}

\maketitle
\hrule{}


\section*{Abstract}

This research paper is aimed to explore a rarely talked about but significantly important aspect of our Indian education system, i.e., inclusion of entrepreneurial education in school curriculums. Entrepreneurship is an exponentially growing realm, especially in India. In fact, this decade is termed to be the era of Indian entrepreneurs. But the underlying problem is that great entrepreneurs are not made in a day. It is a result of endless trials, failures, reattempts and so on. So, the earlier you start, the better. And this is exactly where entrepreneurial education comes to play. Entrepreneurial education can prove to be a groundbreaking innovation in the field of business, technology and overall economic growth. Specifically because we are focusing on nurturing some very practical and creative values to the bright young minds of our country.


\section*{Introduction}

“Entrepreneurship Education”-This is no longer a far-fetched domain for students. Who are the entrepreneurs anyway? The best explanation is someone who can turn ideas into viable plans and strive to realize those plans. Fostering student entrepreneurship can bring significant benefits to their future careers. It motivates students to explore education beyond the walls of the classroom and acquire the skills to drive their ideas.

The primary importance of entrepreneurship is the creation of new job opportunities, innovation, and economic growth. Passion is the first and most important quality of a successful entrepreneur. As a result of this absolute passion, entrepreneurs begin to become more motivated as they take on new tasks and learn new things to achieve their goals. This is also the very reason for inclusion of entrepreneurship in the school curriculum. It empowers them to work for their passion. As schools and universities seek ways to include entrepreneurship education in their curriculum, more and more training academies and institutions are creating new programs for students.


\section*{Literature review}

Real World, Real Skills
The main focus of entrepreneurship education is to teach important life skills that enhance the ability of students to work in the real life. Students can learn a wide range of skills, from collaboration and teamwork to public speaking, presentations, and data analysis. All of these life skills that define an individual's personality can never be learned in traditional books or classroom learning.(1) Many training academies use entrepreneurship training in a filtered way, focusing on key areas such as sales training, marketing know-how, and e-commerce.

Advantages of including entrepreneurial education:

\begin{enumerate}

\item Prepares the students for the uncertain future: We live in the age of unprecedented global and technical transformation. Today's students have a non-secure future with complex global, social and ecological problems. From the perspective of the future global economic forum of the World Economic Forum, half of today's work activities are automated by 2055. Machines and robots will take over. Therefore, we cannot predict what our students need to know after graduation. An entrepreneurial-focused program provides students with important life skills to help them navigate this uncertain future. These skills include learning to accept failure as part of problem solving, teamwork, empathy, and the growth process.(2)


\item Teaches problem identification: Students need to learn to identify a problem before they can learn to solve it. Problem solving has been taught in schools for decades, but it is not the same when it comes to problem identification. Traditionally, problem solving is taught by presenting students with problems that have already been clearly defined by someone else. In the real world, problems can only be resolved if they are correctly identified and explained. Entrepreneurship teaches children to recognize problems they have never encountered. Students can develop problem-solving skills through years of practice, but other successful entrepreneurs are able to recognize problems before they occur and take the necessary precautions to address them. 

\item Nurtures future leaders: Students interested in making a difference will certainly succeed in paving the way. As students learn about entrepreneurial skills at a young age, they tend to integrate new skills and start thinking like leaders. Leadership qualities that are particularly beneficial to female entrepreneurs will force female entrepreneurs to create their own identities by narrowing the gender gap that exists in the corporate world.

\item Improving Creativity: Creative people always tackle problems in different ways, which makes all the difference. Entrepreneurship is strong enough to shape students into more capable individuals and confront the reality of the outside world by promoting creativity, innovation and collaboration. Aside from degrees and qualifications, students will have the very necessary experience to embark on a market journey with a strong foundation. Plus, with the increasing automation and declining job opportunities; creative thinking can help one stay ahead of the curve. Entrepreneurship is an important aspect of a bright future because it is independent of domains and disciplines.

\item Self-development: In the current economic situation, knowledge about academic subjects is no longer enough for new graduates. Students need to improve employment skills and skills, such as obtaining and dealing with information. Entrepreneurs Education and Training offers the ability to recognize the commercial opportunity, self-esteem, knowledge and skills that act. Occasionally recognition, conceptual commercialization, management resources, and lessons for starting business companies are included. It also includes education in traditional business areas such as management, marketing, information systems, and finance.(3)

\item Encourages economic growth: Being an entrepreneur or a self-employed person is and will continue to be an increasingly important factor in economic growth and development. With rising unemployment in a growing world, students need to be taught to create employment opportunities to empower them to become financially independent. Despite the pandemic crisis, Indian start-ups created around two lakh new employment opportunities in 2021. This is the highest number in the last four years.

\item Make the world a better place to live: Entrepreneurs seek to solve problems, meet their needs, and mitigate problems with the help of products and services. You are been wired to make a difference and make the world a better place. The mindset of an entrepreneur is quite different from that of the average person. We need to teach our students that they are the only ones responsible for their future. By participating in the entrepreneurial education, students are not only ready to shape their future, but also to change the world.

\end{enumerate}

\section*{Current Scenario}

As society evolves with technology and innovation, K12 schools are in a stagnant scenario. Education, directly or indirectly, is the driving force behind the economies of all countries. Indeed, many schools have begun to work in groups to adapt to modernization, solve problems, learn online, and integrate science and art. However, graduated students still find that they lack advanced skills and innovative thinking to meet the challenges of the modern workplace. Entrepreneurship courses are currently offered as electives for final year students at most institutions. The example of a fast-growing business school offering it as an essential core course for final-year students was a notable change. 


\section*{A planned approach to entrepreneurial education}

The literature suggests that, despite the fact that the inclination to pursue entrepreneurship is relatively sturdy in India, the academic assist for its improvement remains a miles cry from the agenda. Entrepreneurship nonetheless has an extended manner to head in phrases of incomes the reputation of a desired path amongst college students in India. Perhaps, this reputation acts as motive sufficient to best provide entrepreneurship as an extra-curricular or co-curricular software within side the majority of the faculties and universities in India. The number one impediment to entrepreneurial education in India are lack of institutionalization, indigenous experience and skilled teachers. The Short-time focus on results is another hindrance. Also, in maximum cases, this subject is not considered as core. 
At school, entrepreneurship education is about practical skills, solving current problems, and developing innovative approaches. Meanwhile, college students can learn about creativity by using the variety of tools and online resources available and leveraging social media platforms. Entrepreneurship education planning includes in-depth learning about product development, sales funnels, and business proposal making, and presenting ideas for a particular niche to potential investors. The scope of entrepreneurship education can be adapted according to age as needed. College and college students expect more practical resources, but schools can focus on practicing core entrepreneurship that can be deepened later. Therefore, it is very important to include entrepreneurship, the ability to think creatively and ambitiously, as well as start a business, in the education curriculum. 



\section*{Conclusion}

It is undeniable that entrepreneurship education remains a foreign concept for many schools and universities in India. This is just the right time for institutions to take a step towards improving their brand as well as influencing the quality of their students' education. Demand is expected to grow exponentially as more parents realize the benefits of teaching their children and young adults about entrepreneurship. This is not only affecting the educational quality for students, but also is the appropriate time to take measures to improve their brands. In emerging economies like India, there is an urgent need to develop and promote effective education systems for indigenous entrepreneurship.  In addition, the synergies between entrepreneurship and other core business majors such as marketing are a framework for developing entrepreneurship as a core course for business students to ensure an integrated learning platform.  The framework for building this effective ecosystem for entrepreneurship education is certainly a time consuming need, and a greater focus on knowledge creation is needed to support the framework.  Therefore, this attempt to develop a practical framework for the Indian entrepreneurship ecosystem, backed by preliminary contributions and evidence, is conceptual for further testing the concept and making improvisational designs.

\section*{References}

\begin{enumerate}

\item Benefits of Entrepreneurship Education for Students-TheHigherEducationReview.

\item Five Benefits of Entrepreneurship Education to Students- Marlborough News. 

\item Importance and Benefits of Entrepreneurship to Students. 

\item Basu R. Entrepreneurship Education in India: A Critical Assessment and a Proposed Framework. Vol. 4, Technology Innovation Management Review. 2014. p. 5–10. 

\end{enumerate}



\end{document}
